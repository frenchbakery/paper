\documentclass[conference]{IEEEtran}
\IEEEoverridecommandlockouts
% The preceding line is only needed to identify funding in the first footnote. If that is unneeded, please comment it out.
\usepackage{cite}
\usepackage{amsmath,amssymb,amsfonts}
\usepackage{algorithmic}
\usepackage{graphicx}
\usepackage{textcomp}
\usepackage{xcolor}
\usepackage{caption}
\def\BibTeX{{\rm B\kern-.05em{\sc i\kern-.025em b}\kern-.08em
    T\kern-.1667em\lower.7ex\hbox{E}\kern-.125emX}}

\begin{document}

\title{Programmer friendly communication framework}

\author{
    \IEEEauthorblockN{1\textsuperscript{st} Lilo Zobl}
    \IEEEauthorblockA{
        \textit{Botball Team FrenchBakery} \\
        \textit{HTL Anichstraße}\\
        Innsbruck, Austria \\
        lzobl@tsn.at
    }
    \and
    \IEEEauthorblockN{2\textsuperscript{nd} Matteo Reiter}
    \IEEEauthorblockA{
        \textit{Botball Team FrenchBakery} \\
        \textit{HTL Anichstraße}\\
        Innsbruck, Austria \\
        mareiter@tsn.at
    }
}

\maketitle

\begin{abstract}
    This document will focus on an important topic in software development: Network communication. It will provide an insight into why what network communication is needed for and how it is typically realized. Additionally it will cover problems and annoyances with currently available communication frameworks for certain applications as well as introduce concepts and a possible solution for the before-mentioned. 
\end{abstract}

\section{Introduction}

When it comes to computer 
... Why do we need communication? What for?

IP Networks erwähnen 

TCP/UDP Erwähnen und unterschied (Stream vs. Datagram).

wir behandeln hier nur direkte punkt-zu-punkt verbindungen 

\subsection{Communication frameworks}

... Why do we need communication frameworks? 

using TCP/UDP directly is tedious for many modern application, ...

Job of cf/library?

\begin{itemize}
    \item abstrahieren
    \item benutzerfreundlichkeit bzw programmiererfreundlichkeit
    \item added features for specialized use cases
\end{itemize}


\section{State of the Art}

\begin{itemize}
    \item TCP \& UDP (mostly TCP) for low-level, very performant systems
    \item WebSockets: Implements an additional protocol on top of TCP re-introducing the concept of messages. For Web applications already using HTTP(s)
    \item SocketIO: Builds on top of WebSocket, allows sending javascript events over the network, adding event listeners/emitters and allow sending arbitrary data.
\end{itemize}

\section{Problems with current offer}

This section builds upon the last mentioned protocol SocketIO.

No reassurance of data validity. before using incoming data, one ideally has to check that the structure and type of the received data matches what is expected and handle the error if that isn't the case. This is tedious to do manually for every single event. Additionally since SocketIO allows sending arbitrary data, it is easy to accidentally send data in an incorrect structure, such as accidentally making a typo in an attribute name.






\section{Solution and Design Overview}

\section{Implementation}

\section{Results and Applications}

\section{Limitations}

\section{Conclusion}




\listoffigures

\begin{thebibliography}{00}

    \bibitem{b1} Proof for NordPass study, URL:https://tech.co/password-managers/how-many-passwords-average-person

\end{thebibliography}

\end{document}
